\subsection*{Vize}
Cílem webové aplikace je poskytovat uživatelům prostředí pro přehrávání hudby.
Uživatel si vytvoří svůj osobní účet, který mu umožní nejen poslouchat
hudbu svých oblíbených umělců, ale také vytvářet
vlastní playlisty nebo hodnotit jednotlivé skladby.

Dále poskytujeme možnost registrovat se jako umělec,
což uživatelům umožňuje aktivně přispívat do hudební komunity.
Umělci mohou přidávat vlastní skladby a vytvářet hudební alba.

Aplikace má také uživatelům usnadňovat vyhledávání nových sklade, alb
a poskytovat informace o jednotlivých umělcích.


\subsection*{Role}
\textbf{Uživatel} je osoba, která vytváří a spravuje svůj osobní účet.
Po registraci získává možnost poslouchat hudbu, vytvářet vlastní playlisty
a hodnotit jednotlivé skladby.
Uživatelé mohou využívat funkce jako je získávat informace o umělcích.


\textbf{Umělec} je uživatel s registrovaným účtem, který má zájem aktivně přispívat do hudební komunity.
Jako umělec je brána jak jednotlivá osoba, tak hudební skupina, která vydává skladby pod svým jménem a chová
se tedy stejně jako samostatná osoba umělce.
Může přidávat vlastní skladby a vytvářet hudební alba, která jsou dostupná pro poslech ostatním uživatelům.
Umělci mají možnost prezentovat svou tvorbu a budovat svůj profil.

\subsection*{Vstupy}
\textbf{Uživatelé} jsou klíčovou entitou v projektu.
Registrací získávají možnost poslouchat hudbu, tvořit vlastní playlisty a hodnotit jednotlivé skladby.
Jejich profily obsahují informace jako jméno, příjmení, kontaktní e-mail, zabezpečené heslo, premium status a datum registrace.


Další důležitou entitou jsou \textbf{Umělci}, kteří mají svůj profil na platformě.
Každý umělec má jméno, místo původu a krátkou biografii.
Jsou zodpovědní za přidávání nových skladeb a vytváření alb, která jsou dostupná pro poslech ostatním uživatelům.


Entita \textbf{Skladba} obsahuje detaily o jednotlivých písních.
Každá skladba má svůj název, délku, datum vydání, text (pokud existuje) a spojení s umělci, kteří na ní pracovali.
Skladby mohou být součástí alb.
Uživatelé mohou hodnotit skladby.


Dvě další entity jsou \textbf{Playlisty} a \textbf{Alba}.
Playlisty jsou vytvářeny uživateli a mohou obsahovat různé skladby z různých alb a umělců.
Alba jsou kolekcemi skladeb vytvořenými umělci, která jsou k dispozici pro poslech ostatním uživatelům.
Každé album může obsahovat skladby od jednoho nebo více umělců.


Kromě toho, existují entity jako \textbf{Hodnocení}.
Hodnocení zaznamenává hodnocení skladby uživateli, což umožňuje vyhodnocení oblíbenosti skladeb.

\clearpage

\subsection*{Výstupy}
Pro umělce je k dispozici například výstup, který umožňuje zobrazit jeho skladby vytvořené v určitém období a setřídit je podle hodnocení.
Umělec může specifikovat časové období, například poslední rok nebo konkrétní měsíc, a získat seznam svých skladeb v tomto období.
Tyto skladby budou seřazeny podle hodnocení, kde nejlépe hodnocené skladby budou na vrcholu seznamu.
To umožní umělci zhodnotit, jaké skladby z daného období měly největší ohlas u posluchačů.


Uživatelé mají možnost vyhledávat nejnovější hudební alba na základě data jejich přidání a současně je filtrovat třeba podle svých oblíbených umělců.
Tato funkce jim umožňuje zadat specifické datum a zobrazit alba, která byla přidána od té doby, a to ještě více omezit na dané umělce.


Pro administrátora je k dispozici třeba přehled týkající se uživatelů a plateb za prémiové služby.
Administrátor má možnost sledovat počet nově zaregistrovaných uživatelů v různých časových obdobích.
Dále může získat informace o počtu uživatelů, kteří zakoupili prémiové členství, a sledovat trendy růstu nebo poklesu v registracích či nákupu prémiového členství.

\subsection*{Funkce}
Kolaborativní tvorba playlistů umožňuje uživatelům společně vytvářet a úpravovat playlisty s ostatními uživateli.
Tato funkce podporuje sdílení playlistů a spolupráci na jejich obsahu a úpravách.


Pro zajištění kvality obsahu playlistů a minimalizaci chyb systém provádí kontrolu unikátnosti skladeb.
Při přidání nové skladby do playlistu ověřuje, zda již skladba v playlistu není obsažena, přičemž upozorní
uživatele na možnost duplicity a umožní mu potvrzení nebo zrušení přidání.
Taktéž systém monitoruje dostupnost skladeb v playlistu a v případě, že některá skladba není momentálně dostupná,
nabídne uživateli možnost nahrazení jinou skládankou.


Dále zabezpečuje integritu playlistů během společné editace, takže při práci více uživatelů
zároveň se snaží zabránit duplicitním změnám a konfliktním situacím.
Tímto způsobem systém uživatelům umožňuje vytvářet obsáhlé a rozmanité playlisty díky společné tvorbě a úpravám obsahu,
zatímco zároveň minimalizuje možnost výskytu chyb a duplikací skladeb.
\clearpage