\documentclass[a4paper, 10pt, oneside]{article}       % nechceme twoside
\usepackage{lmodern}                                  % latin modern font
\usepackage[utf8]{inputenc}                           % UTF-8
\usepackage[a4paper, left=20mm, right=30mm, top=30mm, bottom=35mm]{geometry}
\usepackage{booktabs}
\usepackage[table]{xcolor}
\usepackage{xcolor}
\usepackage{multirow}
\usepackage{siunitx}
\usepackage{listings}
\usepackage{caption}

\usepackage[ruled, vlined, czech] {algorithm2e}
\usepackage[czech=quotes]{csquotes}
\usepackage{cmap}
\usepackage{graphicx}
\usepackage[T1]{fontenc}
\usepackage[style=numeric,backend=biber,sorting=nty]{biblatex}
\usepackage[czech]{babel}                             % tenhle musi byt za biblatex

\usepackage{graphicx}
\usepackage[colorlinks=true, allcolors=blue]{hyperref}
\usepackage{verbatim}
\usepackage{algorithm2e}

\usepackage{amsmath}
\usepackage{amssymb}                                  % velice důležité pro symboly

\newcolumntype{e}{D{,}{,}{3.5}}                       % zarovnání hodnot v tabulce podle .

% Úvodní strana
\usepackage{titling}
\title{\Huge\textbf{Diskrétní matematika}\\
{\textbf{Projekt}}\\
{\Large číslo zadání \underline{\makebox[1cm]{X}}}}
\author{}                                             % nutno ponechat prázdné
\date{}
\renewcommand\maketitlehooka{\null\mbox{}\vfill}      % vertically CENTER title page!!!
\renewcommand\maketitlehookd{\vfill\null}

\usepackage{fancyhdr}
\renewcommand{\headrulewidth}{0pt}                    % odstraní čáru pod textem v záhlaví

%\usepackage[useregional]{datetime2}                  % pro hod:min pomocí \DTMnow místo \today
\pagestyle{fancy} 				                      % nastavení stylu záhlaví a zápatí
\fancyhead[LO]{\Large VŠB - TUO}
\fancyhead[RO]{\Large Datum \underline{\the\day.\the\month.\the\year}}
\fancyfoot[LO]{\Large Osobní číslo \underline{\makebox[2.5cm]{END0155}}}
\fancyfoot[RO]{\Large Jméno \underline{\makebox[4cm]{Jméno Příjmení}}}
\fancyfoot[C]{}                                       % smaže číslování titulní strany!!!

% ZAČÁTEK*********************************************
\begin{document}

\shorthandoff{-}

% TITULNÍ STRANA
\maketitle\thispagestyle{fancy}

\vfill
\hfill
\begin{table}[b]
  \begin{tabular}{c|p{3cm}}
    \textbf{Příklad} & \textbf{Poznámky} \\
    \hline
        \rule{0pt}{1cm} \centering 1 &  \\
        \rule{0pt}{1.5cm}\\
        \rule{0pt}{1cm} \centering 2 &  \\
        \rule{0pt}{1.5cm}\\
  \end{tabular}
\end{table}

\clearpage                                            % zprovozní náš custom header!!!

% Záhlaví a zápatí
\fancyhead{}                                          % clear all header fields
\fancyfoot{}                                          % clear all footer fields
%\fancyhf{}                                           % clear existing header/footer entries

\fancyhead[LO]{\textit{Diskrétní matematika}}         % text v záhlaví

\fancyhead[RO]{\thepage}                              % číslování v záhlaví

\newpage
\setcounter{page}{1}
\section{Kombinatorika nebo Kongruence (záleží dle zadání)}\label{sec:sec_1}

\textit{Zaměstnanec depa má za úkol natankovat 3 stejné kamiony. První kamion tankoval pouze plnými dvoulitrovými kanystry, přičemž mu po naplnění nádrže zbyl v posledním kanystru jeden litr nafty. Další dva kamiony tankoval pouze plnými třílitrovými kanystry, přičemž mu po naplnění nádrží zbyly v posledním kanystru dva litry nafty. Zaměstnanec si vzpomněl, že když naposled tankoval tyto tři kamiony, používal pouze pětilitrové kanystry a na konci mu v posledním použitém kanystru zbyl jeden litr nafty.\\
a) Najděte všechny přípustné velikosti nádrže kamionu, které odpovídají zadání.\\
b) Jaké mohly být velikosti nádrže kamionu, pokud víme, že nebyly větší než 500 litrů? Úlohu řešte jako soustavu kongruencí.}
\newline\\
\textbf{Máme celkem 3 nejlepší způsoby psaní matematických výrazů:}

$$2023x\equiv 155\pmod{13},\quad k\in\mathbb{N}$$
bez odstřelení na nový řádek $365x \equiv 12 \pmod{0}$
\[33x\equiv 11\pmod{77}\implies 3x=101010\]

\newpage
\section{Teorie grafů}\label{sec:sec_2}

\textit{Mějme kompletní graf $K_{\text{n}}$ a označme vrcholy čísly 1 až n.\\
a) Kolik má kompletní graf $K_{\text{n}}$ faktorů?\\
b) Kolik existuje různých podgrafů tohoto grafu? Izomorfní grafy považujeme za různé, pokud mají různá očíslování vrcholů. Svůj postup dobře okomentujte.\\
c) Obecné řešení vyčíslete pro kompletní grafy $K_{\text{3}}$ a $K_{\text{6}}$.}

\end{document} % konec celého dokumentu