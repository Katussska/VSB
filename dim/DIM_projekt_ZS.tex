\documentclass[a4paper, 10pt, oneside]{article}       % nechceme twoside
\usepackage{lmodern}                                  % latin modern font
\usepackage[utf8]{inputenc}                           % UTF-8
\usepackage[a4paper, left=20mm, right=30mm, top=30mm, bottom=35mm]{geometry}
\usepackage{booktabs}
\usepackage[table]{xcolor}
\usepackage{multirow}
\usepackage{siunitx}
\usepackage{listings}
\usepackage{caption}

\usepackage[ruled, vlined, czech] {algorithm2e}
\usepackage[czech=quotes]{csquotes}
\usepackage{cmap}
\usepackage{graphicx}
\usepackage[T1]{fontenc}
\usepackage[style=numeric,backend=biber,sorting=nty]{biblatex}
\usepackage[czech]{babel}                             % tenhle musi byt za biblatex
\usepackage[colorlinks=true, allcolors=blue]{hyperref}
\usepackage{verbatim}
\usepackage{amsmath}
\usepackage{amssymb}                                  % velice důležité pro symboly

\newcolumntype{e}{D{,}{,}{3.5}}                       % zarovnání hodnot v tabulce podle .

% Úvodní strana
\usepackage{titling}
\title{\Huge\textbf{Diskrétní matematika}\\
{\textbf{Projekt}}\\
{\Large číslo zadání \underline{\makebox[1cm]{1}}}}
\author{}                                             % nutno ponechat prázdné
\date{}
\renewcommand\maketitlehooka{\null\mbox{}\vfill}      % vertically CENTER title page!!!
\renewcommand\maketitlehookd{\vfill\null}

\usepackage{fancyhdr}
\renewcommand{\headrulewidth}{0pt}                    % odstraní čáru pod textem v záhlaví

%\usepackage[useregional]{datetime2}                  % pro hod:min pomocí \DTMnow místo \today
\pagestyle{fancy} 				                      % nastavení stylu záhlaví a zápatí
\fancyhead[LO]{\Large VŠB - TUO}
\fancyhead[RO]{\Large Datum \underline{\the\day.\the\month.\the\year}}
\fancyfoot[LO]{\Large Osobní číslo \underline{\makebox[2.5cm]{BAI0033}}}
\fancyfoot[RO]{\Large Jméno \underline{\makebox[4cm]{Kateřina Baierová}}}
\fancyfoot[C]{}                                       % smaže číslování titulní strany!!!

% ZAČÁTEK*********************************************
\begin{document}

\shorthandoff{-}

% TITULNÍ STRANA
\maketitle\thispagestyle{fancy}

\vfill
\hfill
\begin{table}[b]
  \begin{tabular}{c|p{3cm}}
    \textbf{Příklad} & \textbf{Poznámky} \\
    \hline
        \rule{0pt}{1cm} \centering 1 &  \\
        \rule{0pt}{1.5cm}\\
        \rule{0pt}{1cm} \centering 2 &  \\
        \rule{0pt}{1.5cm}\\
  \end{tabular}\label{tab:table}
\end{table}

\clearpage                                            % zprovozní náš custom header!!!

% Záhlaví a zápatí
\fancyhead{}                                          % clear all header fields
\fancyfoot{}                                          % clear all footer fields
%\fancyhf{}                                           % clear existing header/footer entries

\fancyhead[LO]{\textit{Diskrétní matematika}}         % text v záhlaví

\fancyhead[RO]{\thepage}                              % číslování v záhlaví

\newpage
\setcounter{page}{1}
\section{Kombinatorika}\label{sec:sec_1}

\textit{Babička peče sušenky. Když se je pokusila vyskládat na plech v řadách po čtyřech kusech, tak jí
nějaké zbyly. Když se je pokusila vyskládat na plech v řadách po pěti kusech, tak jich zbylo
dvakrát tolik. Babička si poté všimla, že pokud připeče šest sušenek navíc, tak jí ani v jednom z
předchozích vyskládaní nezbude žádná sušenka navíc.\\
a) Najděte všechna přípustná množství sušenek, které odpovídají zadání.\\
b) Kolik sušenek mohla mít původně napečených, pokud víme, že těsta měla maximálně na 50.\\
c) Kolik nejméně jich mohlo původně při vyskládání po čtyřech zbýt?\\}
\newline\\

$$2023x\equiv 155\pmod{13},\quad k\in\mathbb{N}$$
bez odstřelení na nový řádek $365x \equiv 12 \pmod{0}$
\[33x\equiv 11\pmod{77}\implies 3x=101010\]

\begin{center}
    $$x\equiv \;?\pmod{4}$$
    $$x\equiv \;?\pmod{5}$$
    \rule{5cm}{0.4pt} \\
    \textcolor{darkgray}{Při přidání 6 sušenek nebude nic chybět a nebude nic navíc}
    $$x + 6\equiv \;0\pmod{4}$$
    $$x + 6\equiv \;0\pmod{5}$$
    \rule{5cm}{0.4pt}
    $$x\equiv \;-6\pmod{4}$$
    $$x\equiv \;-6\pmod{5}$$
    \rule{5cm}{0.4pt}\\
    \textcolor{darkgray}{x \equiv 4n - 6 \;odvodíme \;a \;přidáme \;do \;(mod \;5)}\\
    $$x \equiv -6 \pmod{4} \implies x \equiv 4n - 6$$
\end{center}



\newpage
\section{Teorie grafů}\label{sec:sec_2}

\textit{Mějme libovolný souvislý graf se stupňovou posloupností (3,3,2,2,2,2,2,2,2,2), o kterém navíc víme,
    že mezi každou dvojicí vrcholů ve vzdálenosti 3 a více existuje právě jedna cesta.\\
a) Může být takový graf hamiltonovský? Pokud ano, nakreslete ho. Pokud ne, pečlivě zdůvodněte proč.\\
b) Nakreslete graf se stejnou stupňovou posloupností, který je hamiltonovský.}

\end{document} % konec celého dokumentu